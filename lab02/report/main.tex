\documentclass[a4paper, 11pt]{article}
\usepackage[english]{babel}
\usepackage{fullpage}

\usepackage[pdftex]{graphicx}
\graphicspath{{../pdf/}{../jpeg/}}
\DeclareGraphicsExtensions{.pdf,.jpeg,.png}

% ----------------------------------------------

% Definitions of languages: --------------------
\usepackage{listings}
\lstdefinestyle{cStyle}{
  basicstyle=\scriptsize,
  breakatwhitespace=false,
  breaklines=true,
  captionpos=b,
  keepspaces=true,
  numbersep=5pt,
  showspaces=false,
  gobble=4,
  tabsize=4,
  showstringspaces=false,
  showtabs=false,
}
\renewcommand*{\lstlistingname}{Code}

% ----------------------------------------------

\begin{document}

\noindent
\large\textbf{CES-27 Distributed Processing} \\
\textbf{2nd Activity} \\
\normalsize Prof Hirata and Prof Juliana  \\
Carlos Matheus Barros da Silva \hfill September 2019

\section*{2th Activity}

It was implemented the Ricart-Agrawala's algorithm. The implementation was made in Go and it can be seen on the Repository\footnote{https://github.com/CarlosMatheus/CES-27/tree/master/lab02}.

\section*{First Case}

As It was suggested, it was conducted a test with 3 terminal windows, on two of them were open the process and on the third one was open the shared process.

The test was made according to the model represented on Figure \ref{img_task1}. The results can be seen on the terminal windows shown from Figure \ref{img_task1_example_window1} to Figure \ref{img_task1_example_window3}.

\begin{figure}[h]
  \begin{center}
  \includegraphics[width=4in]{./imgs/case1.png}
  \caption{Model representing the execution of Case 1.}
  \label{img_task1}
  \end{center}
\end{figure}

\begin{figure}[h]
  \begin{center}
  \includegraphics[width=4.5in]{./imgs/case1process1.png}
  \caption{Process 1 after execution of Case 1 example test case.}
  \label{img_task1_example_window1}
  \end{center}
\end{figure}

\begin{figure}[h]
  \begin{center}
  \includegraphics[width=4.5in]{./imgs/case1process1.png}
  \caption{Process 2 after execution of Case 1 example test case.}
  \label{img_task1_example_window2}
  \end{center}
\end{figure}

\begin{figure}[h]
  \begin{center}
  \includegraphics[width=4.5in]{./imgs/case1CS.png}
  \caption{Shared Resource after execution of Case 1 example test case.}
  \label{img_task1_example_window3}
  \end{center}
\end{figure}

As expected, the mutual exclusion woked; thus, when a process request access the other reply allowing the access when it is on \textit{released} state, then the first process accesses the shared resource. When it finishes, it returns to released state. Then the the second process do the same steps. All the logic worked as well as the logical clocks on each process, therefore the simulation on the terminals matched the model representd on Figure \ref{img_task1}.

\section*{Second Case}

It was built a test case with 6 terminal windows. The test was made according to the model represented on Figure \ref{img_task2}. The results can be seen on the terminal windows shown from Figure \ref{img_task2_example_window1} to Figure \ref{img_task2_example_window6}.

\begin{figure}[h]
  \begin{center}
  \includegraphics[width=4in]{./imgs/case2.png}
  \caption{Model representing the execution of Case 1.}
  \label{img_task2}
  \end{center}
\end{figure}

\begin{figure}[h]
  \begin{center}
  \includegraphics[width=4.5in]{./imgs/case2process1.png}
  \caption{Process 1 after execution of Case 2 example test case.}
  \label{img_task2_example_window1}
  \end{center}
\end{figure}

\begin{figure}[h]
  \begin{center}
  \includegraphics[width=4.5in]{./imgs/case2process2.png}
  \caption{Process 2 after execution of Case 2 example test case.}
  \label{img_task2_example_window2}
  \end{center}
\end{figure}

\begin{figure}[h]
  \begin{center}
  \includegraphics[width=4.5in]{./imgs/case2process3.png}
  \caption{Process 3 after execution of Case 2 example test case.}
  \label{img_task2_example_window3}
  \end{center}
\end{figure}

\begin{figure}[h]
  \begin{center}
  \includegraphics[width=4.5in]{./imgs/case2process4.png}
  \caption{Process 4 after execution of Case 2 example test case.}
  \label{img_task2_example_window4}
  \end{center}
\end{figure}

\begin{figure}[h]
  \begin{center}
  \includegraphics[width=4.5in]{./imgs/case2process5.png}
  \caption{Process 5 after execution of Case 2 example test case.}
  \label{img_task2_example_window5}
  \end{center}
\end{figure}

\begin{figure}[h]
  \begin{center}
  \includegraphics[width=4.5in]{./imgs/case2CS.png}
  \caption{Shared Resource after execution of Case 2 example test case.}
  \label{img_task2_example_window6}
  \end{center}
\end{figure}

As expected, the mutual exclusion woked; Even when multiple process were wanting to access the shared resource at the same time, the mutual exclusion garanteed the safity and the lifeness of the operation. All the logic worked as well as the logical clocks on each process, therefore the simulation on the terminals matched the model representd on Figure \ref{img_task1}.


\end{document}
